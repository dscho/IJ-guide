%Bibliography
\renewcommand{\refname}{ImageJ Related Publications}
\phantomsection
\addcontentsline{toc}{section}{\refname}

% Allow a small paragraph before listing the bibtex file
\let\thebibliographyOLD\thebibliography
\renewcommand{\thebibliography}[1]{\thebibliographyOLD{#1}%
  \item[]
  \hskip-\leftmargin \begin{minipage}{\textwidth}


\label{sec:IJ-related-pub}

The following references are a small sample (particularly biased towards
the life sciences) of the bibliography directly related to ImageJ,
the standard in scientific image analysis. These publications  include:
1) technical articles and books describing routines implemented in
ImageJ, 2) research articles that have made extensive use of ImageJ
as a scientific tool or 3) reviews that discuss ImageJ pertinently.
A similar list is maintained on the \nameref{sub:Fiji-intro} \href{http://fiji.sc/wiki/index.php/Publications}{website}.

\medskip{}


To reference ImageJ one of the \href{http://imagej.nih.gov/ij/docs/faqs.html\#cite}{following citations}
is possible:
\begin{enumerate}
\item Rasband, W.S., ImageJ, U.S.\ National Institutes of Health, Bethesda,
Maryland, USA, \href{http://imagej.nih.gov/ij/}{imagej.nih.gov/ij/},
1997--2012.
\item Abr�moff, M.D., Magalh�es, P.J.\ and Ram, S.J. \emph{Image Processing
with ImageJ}. Biophotonics International, 11(7):36--42, 2004 (\href{http://webeye.ophth.uiowa.edu/dept/biograph/abramoff/imagej.pdf}{PDF})
\cite{Abramoff:2004p4386}.
\end{enumerate}
To reference this document:
\begin{itemize}
\item Ferreira T.\ and Rasband W\emph{. }ImageJ User Guide --- IJ\,1.46,
\href{http://imagej.nih.gov/ij/docs/guide/}{imagej.nih.gov/ij/docs/guide/},
2010--2012
\end{itemize}
\medskip{}


  \end{minipage}\bigskip}
\small{}

\bibliographystyle{plainurl}
\nocite{*}
\bibliography{ImageJRefs}

