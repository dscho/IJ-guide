
\part{Getting Started\label{part:Getting-Started}}

This part provides basic information on ImageJ installation, troubleshooting
and update strategies. It discusses \nameref{sub:Fiji-intro}\nomenclature{FIJI}{Fiji Is Just ImageJ}
and \nameref{sub:ImageJ2intro} as well as third-party software related
to ImageJ. Being impossible to document all the capabilities of ImageJ
without exploring technical aspects of image processing, external
resources allowing willing readers to know more about digital signal
processing are also provided.


\section{Introduction\label{sec:What-is-ImageJ?}}

ImageJ is a \href{http://rsb.info.nih.gov/ij/disclaimer.html}{public domain}
Java image processing and analysis program inspired by \href{http://rsb.info.nih.gov/nih-image/}{NIH Image}
for the Macintosh. It runs, either as an online applet or as a downloadable
application, on any computer with a Java\,1.5 or later virtual machine.
\href{http://imagej.nih.gov/ij/download.html}{Downloadable distributions}
are available for Windows, Mac OS\,X and Linux. It can display, edit,
analyze, process, save and print 8--bit, 16--bit and 32--bit images.
It can read many image formats including TIFF, GIF, JPEG, BMP, DICOM,
FITS and `raw'. It supports `stacks' (and hyperstacks), a series
of images that share a single window. It is multithreaded, so time-consuming
operations such as image file reading can be performed in parallel
with other operations%
\footnote{A somehow outdated list of ImageJ's features is available at \url{http://imagej.nih.gov/ij/features.html}%
}.

It can calculate area and pixel value statistics of user-defined selections.
It can measure distances and angles. It can create density histograms
and line profile plots. It supports standard image processing functions
such as contrast manipulation, sharpening, smoothing, edge detection
and median filtering.

It does geometric transformations such as scaling, rotation and flips.
Image can be zoomed up to 32\,:\,1 and down to 1\,:\,32. All analysis
and processing functions are available at any magnification factor.
The program supports any number of windows (images) simultaneously,
limited only by available memory.

Spatial calibration is available to provide real world dimensional
measurements in units such as millimeters. Density or gray scale calibration
is also available.

ImageJ was designed with an open architecture that provides extensibility
via Java plugins. Custom acquisition, analysis and processing plugins
can be developed using ImageJ's built in editor and Java compiler.
User-written plugins make it possible to solve almost any image processing
or analysis problem.

Being public domain open source software, an ImageJ user has the \href{http://wikieducator.org/The_right_license/The_essential_freedoms}{four essential freedoms}
defined by the Richard Stallman in 1986: 1) The freedom to run the
program, for any purpose; 2) The freedom to study how the program
works, and change it to make it do what you wish; 3) The freedom to
redistribute copies so you can help your neighbor; 4) The freedom
to improve the program, and release your improvements to the public,
so that the whole community benefits.

ImageJ is being developed on Mac OS\,X using its built in editor
and Java compiler, plus the \emph{BBEdit} editor and the \emph{Ant}
build tool. The source code is freely \href{http://imagej.nih.gov/ij/developer/source/index.html}{available}.
The author, Wayne Rasband (\href{mailto:wsr@nih.gov}{\nolinkurl{wsr@nih.gov}}),
is a Special Volunteer at the National Institute of Mental Health,
Bethesda, Maryland, USA.


\calso{\href{http://imagejdev.org/history}{History of ImageJ at imagejdev.org}}


\section{Installing and Maintaining ImageJ\label{sec:Updating-ImageJ}\index{Updates}}

ImageJ can be downloaded from \url{http://imagej.nih.gov/ij/download.html}.
Details on how to install ImageJ on \index{Linux}\href{http://imagej.nih.gov/ij/docs/install/linux.html}{Linux},
\href{http://imagej.nih.gov/ij/docs/install/mac.html}{Mac OS 9},
\index{Mac OS X}\href{http://imagej.nih.gov/ij/docs/install/osx.html}{Mac OS X}
and \index{Windows (OS)}\href{http://imagej.nih.gov/ij/docs/install/windows.html}{Windows}
\cite{C-WindowsInstaller} are available at \url{http://imagej.nih.gov/ij/docs/install/}
(\textsf{\userinterface{\textsf{Help\lyxarrow{}\nameref{sub:Installation...}}}}
command). Specially useful are the platform-specific \emph{Troubleshooting}
and \emph{Known Problems }sections. \index{Fiji}\nameref{sub:Fiji-intro}
installation is described at \url{http://fiji.sc/wiki/index.php/Downloads}.

The downloaded package may not contain the latest bug fixes so it
is recommended to upgrade ImageJ right after a first installation.
\index{Updates}Updating IJ\nomenclature{IJ}{ImageJ} consists only
of running \textsf{\userinterface{\textsf{Help\lyxarrow{}\nameref{sub:Update-ImageJ...}}}},
which will install the latest \filenameref{\href{http://imagej.nih.gov/ij/upgrade/}{ij.jar}}
in the ImageJ folder (on Linux and Windows) or inside the ImageJ.app
(on Mac OS\,X). 

\textsf{\userinterface{\textsf{Help\lyxarrow{}\nameref{sub:Update-ImageJ...}}}}
can be used to upgrade (or downgrade) the \filenameref{ij.jar} file
to \emph{release updates} or \emph{daily builds}. Release updates
are announced frequently on the \href{http://imagej.nih.gov/ij/notes.html}{IJ news website}
and are labelled alphabetically (e.g., v.\,1.43m). Typically, these
releases contain several new features and bug fixes, described in
detail on the \href{http://imagej.nih.gov/ij/notes.html}{ImageJ News page}.
\emph{Daily builds,} on the other hand, are labelled with numeric
sub-indexes (e.g., v.\,1.43n4) and are often released without documentation.
Nevertheless, if available, release notes for daily builds can be
found at \url{http://imagej.nih.gov/ij/source/release-notes.html}.
When a release cycle ends (v.\,1.42 ended with 1.42q, v.\,1.43 with
1.43u, etc.) an \emph{installation package }is created, downloadable
from \url{http://imagej.nih.gov/ij/download.html}. Typically, this
package is bundled with a small list of add-ons (\nameref{sub:Macros-ExtendingIJ},
\nameref{sub:Scripts} and \nameref{sub:Plugins}).


\calso{\href{http://imagej.nih.gov/ij/macros/toolsets/Luts\%20Macros\%20and\%20Tools\%20Updater.txt}{Luts, Macros and Tools Updater},
a macro toolset that performs live-updating of \href{http://imagej.nih.gov/ij/macros/}{macros}
listed on the ImageJ web site}


\subsection{ImageJDistributions\label{sec:ImageJ-Distributions}}

ImageJ alone is not that powerful: it's real strength is the vast
repertoire of \nameref{sub:Plugins} that extend ImageJ's functionality
beyond its basic core. The many hundreds, probably thousands,  freely
available plugins from contributors around the world play a pivotal
role in ImageJ's success \cite{Collins:2007p13684}. Running \textsf{\userinterface{\textsf{Help\lyxarrow{}\nameref{sub:Update-ImageJ...}}}},
however, will not update any of the plugins you may have installed%
\footnote{Certain plugins, however, provide self-updating mechanisms (e.g.,
\href{http://simon.bio.uva.nl/objectj/}{ObjectJ} and the \index{LOCI Bio-Formats}\href{http://www.loci.wisc.edu/ome/formats.html}{LOCI Bio-Formats library}).%
}.

ImageJ add-ons (\nameref{sub:Plugins}, \nameref{sub:Scripts} and
\nameref{sub:Macros-ExtendingIJ}) are available from several sources
(\href{http://imagej.nih.gov/ij/plugins/}{ImageJ's plugins page}
{[}\textsf{\userinterface{\textsf{Help\lyxarrow{}\nameref{sub:Plugins...}}}}{]},
\href{http://imagejdocu.tudor.lu/doku.php?id=plugin:start}{ImageJ Information and Documentation Portal}
and \href{http://fiji.sc/wiki/index.php/Category:Plugins}{Fiji's webpage},
among others) making manual updates of a daunting task. This reason
alone, makes it extremely convenient the use of \nameref{sec:ImageJ-Distributions}
bundled with a pre-organized collection of add-ons.

Below is a list of the most relevant projects that address the seeming
difficult task of organizing and maintaining ImageJ beyond its basics.
If you are a life scientist and have doubts about which distribution
to choose you should opt for \nameref{sub:Fiji-intro}. It is heavily
maintained, offers an automatic updater, improved scripting capabilities
and ships with powerful plugins. More specialized adaptations of ImageJ
are discussed in \nameref{sub:Other-Software-Packages}.


\subsubsection*{Fiji\label{sub:Fiji-intro}}

\href{http://fiji.sc/}{Fiji}\index{Fiji} (\emph{Fiji Is Just ImageJ---Batteries
included}) is a distribution of ImageJ together with Java, Java 3D
and several plugins organized into a coherent menu structure. Citing
its developers, ``Fiji compares to ImageJ as Ubuntu compares to Linux''.
The main focus of Fiji is to assist research in life sciences, targeting
image registration, stitching, segmentation, feature extraction and
3D visualization, among others. It also supports many scripting languages
(BeanScript, Clojure, Jython, Python, Ruby, \emph{see} \nameref{sec:ScriptingOtherLang}).
Importantly, Fiji ships with a \href{http://fiji.sc/wiki/index.php/Update_Fiji}{convenient updater}
that knows whether your files are up-to-date, obsolete or locally
modified. \href{http://fiji.sc/wiki/index.php/Documentation}{Comprehensive documentation}
is available for most of its plugins. The Fiji project was presented
publicly for the first time at the \href{http://imagejconf.tudor.lu/doku.php}{ImageJ User and Developer Conference}
in November 2008.


\subsubsection*{MBF\ ImageJ\label{sub:MBFImageJintro}}

The \index{MBF ImageJ}\index{ImageJ for Microscopy@ImageJ for Microscopy  \see{MBF ImageJ,}}\href{http://www.macbiophotonics.ca/imagej/}{MBF ImageJ bundle}
or \emph{ImageJ for Microscopy} (formerly \href{http://www.uhnres.utoronto.ca/facilities/wcif/imagej/}{WCIF-ImageJ})
features a collection of plugins and macros, collated and organized
by Tony Collins at the MacBiophotonics facility, McMaster University.
It is accompanied by a \href{http://www.macbiophotonics.ca/imagej/}{comprehensive manual}
describing how to use the bundle with light microscopy image data.
It is a great resource for microscopists but is not maintained actively,
lagging behind the development of core ImageJ.

Note that you can add plugins from MBF ImageJ to Fiji, combining the
best of both programs. Actually, you can use multiple ImageJ distributions
simultaneously, assemble your own ImageJ bundle by gathering the plugins
that best serve your needs (probably, someone else at your institution
already started one?) or create symbolic links to share plugins between
different installations.


\calso{Description of all ImageJ related projects at \href{http://imagejdev.org/faq\#n141}{ImageDev}}


\subsection{Related Software}


\subsubsection{Software Packages Built on Top of ImageJ\label{sub:Other-Software-Packages}}
\begin{description}
\item [{Bio7}] \href{http://bio7.org/}{Bio7} is an integrated development
environment for \index{Bio7}ecological modeling with a main focus
on individual based modeling and spatially explicit models. Bio7 features:
Statistical analysis (using R); Spatial statistics; Fast communication
between \index{R (GNU S)@R (GNU S)  \see{Interoperability,}}R and
Java; BeanShell and Groovy support; Sensitivity analysis with an embedded
flowchart editor and creation of 3D OpenGL (Jogl) models (\emph{see
also} RImageJ in \nameref{sec:ImageJ-Interoperability}).
\item [{BoneJ}] \href{http://bonej.org/}{BoneJ} \index{BoneJ}is a collection
of tools for trabecular geometry and whole bone shape analysis.
\item [{$\micro$Manager}] \href{http://www.micro-manager.org/}{Micro-Manager}
is a software package for control of automated microscopes. It lets
you execute common microscope image acquisition strategies such as
time-lapses, multi-channel imaging, z-stacks, and combinations thereof.
\index{Micro@$\micro$Manager}$\micro$Manager works with microscopes
from all four major manufacturers, most scientific-grade cameras and
many peripherals used in microscope imaging.
\item [{MRI--CIA}] \href{http://www.mri.cnrs.fr/index.php?m=38}{MRI Cell Image Analyzer},
developed by the Montpellier RIO Imaging facility (CNRS), is a rapid
image analysis application development framework, adding visual scripting
interface to ImageJ's capabilities. It can create batch applications
as well as interactive applications. The applications include the
topics \textquotedblleft{}DNA combing\textquotedblright{}, \textquotedblleft{}quantification
of stained proteins in cells\textquotedblright{}, \textquotedblleft{}comparison
of intensity ratios between nuclei and cytoplasm\textquotedblright{}
and \textquotedblleft{}counting nuclei stained in different channels''.
\item [{ObjectJ}] \href{http://simon.bio.uva.nl/objectj/index.html}{ObjectJ},
the successor of \href{http://simon.bio.uva.nl/Object-Image/object-image.html}{object-image},
supports graphical vector objects that non-destructively mark images
on a transparent layer. Vector objects can be placed manually or by
macro commands. Composite objects can encapsulate different color-coded
marker structures in order to bundle features that belong together.
ObjectJ provides back-and-forth navigation between results and images.
The results table supports statistics, sorting, color coding, qualifying
and macro access.
\item [{SalsaJ}] \href{http://www.euhou.net/index.php?option=com_content&task=view&id=7&Itemid=9}{SalsaJ}
is a student-friendly software developed specifically for the \href{http://www.euhou.net/}{EU-HOU project}.
It is dedicated to image handling and analysis of astronomical images
in the classroom. \index{SalsaJ}SalsaJ has been translated into several
languages.
\item [{\label{misc:TrakEM2}TrakEM2}] \href{http://www.ini.uzh.ch/~acardona/trakem2.html}{TrakEM2}
is a program for morphological \index{Data mining@Data mining \see{TrakEM2,}}data
mining, three-dimensional \index{Modeling@Modeling \see{TrakEM2 and Bio7,}}modeling
and image stitching, registration, editing and annotation \cite{Cardona:2010p18306}.
\index{TrakEM2}\index{Fiji}TrakEM2 is \href{http://fiji.sc/wiki/index.php/TrakEM2}{distributed with Fiji}
and \href{http://www.ini.uzh.ch/~acardona/trakem2_manual.html}{capable of}:\vspace{-8pt}


\begin{description}
\item [{3D\ modeling}] Objects in 3D, defined by sequences of contours,
or profiles, from which a skin, or mesh, can be constructed, and visualized
in 3D.
\item [{Relational\ modeling}] The extraction of the map that describes
links between objects. For example, which neuron contacts which other
neurons through how many and which synapses. 
\end{description}
\end{description}

\calso{\index{BioImageXD}\href{http://www.bioimagexd.net/}{BioImageXD},
\index{Endrov}\href{http://www.endrov.net/}{Endrov}, \index{Image SXM}\href{http://www.liv.ac.uk/\%7Esdb/ImageSXM/}{Image SXM}}


\subsubsection{ImageJ Interoperability\label{sec:ImageJ-Interoperability}}

Several packages exist that allow ImageJ to \index{Interoperability}interact
with other applications/environments:
\begin{description}
\item [{Bitplane\ Imaris}] \href{http://www.bitplane.com/go/products/imarisxt}{ImarisXT}
can load and execute ImageJ plugins. \index{Imaris}\href{http://www.bitplane.com/go/products/imarisxt/xtensions/imagej}{bpImarisAdapter}
(Windows only and requiring valid licenses for Imaris and ImarisXT)
allows the exchange of images between Imaris and ImageJ.
\item [{CellProfiler}] \index{CellProfiler@CellProfiler  \see{Interoperability,}}\href{http://www.cellprofiler.org/}{CellProfiler}
\cite{Carpenter:2006p1986} features \href{http://cellprofiler.org/CPmanual/RunImageJ.html}{RunImageJ},
a module that allows ImageJ plugins to be run in a CellProfiler pipeline.
\item [{Icy}] \href{http://www.bioimageanalysis.org/icy/}{Icy}, an open
source community software for \index{Icy}bio-imaging, executes ImageJ
plugins with almost 100\% plugin compatibility. 
\item [{Knime}] \index{Knime} \href{http://knime.org/}{Knime} (\nomenclature{Knime}{Konstanz Information Miner}Konstanz
Information Miner) contains several image processing nodes (\nomenclature{KNIP}{Knime Image Processing}\href{http://tech.knime.org/community/image-processing}{KNIP}\index{KNIP@KNIP  \see{Knime,}})
that are capable of executing ImageJ plugins and macros.
\item [{Open\ Microscopy\ Environment}] \nomenclature{OME}{Open Microsopy Environment}All
\href{http://www.openmicroscopy.org/}{Open Microscopy Environment}
projects such as \index{LOCI Bio-Formats}\href{http://www.openmicroscopy.org/site/products/bio-formats}{Bio-Formats},
\href{http://www.openmicroscopy.org/site/products/visbio}{VisBio}
and \href{http://www.openmicroscopy.org/site/products/omero}{OMERO}
integrate well with ImageJ.
\item [{RImageJ\ ---\ R\ bindings\ for\ ImageJ}] Bindings between
ImageJ and \index{R (GNU S)@R (GNU S)  \see{Interoperability,}}\href{http://www.r-project.org/}{R (GNU S)}
--- The free software environment for statistical computing and graphics.
The documentation for RImageJ is available at \url{http://cran.r-project.org/web/packages/RImageJ/RImageJ.pdf}
(\emph{see also} Bio7 in \nameref{sub:Other-Software-Packages}).
\item [{MIJ\ ---\ Matlab--ImageJ\ bi-directional\ communication}] A
Java package for bi-directional data exchange between \index{MATLAB@MATLAB  \see{Interoperability,}}Matlab
and ImageJ, allowing to exchange images between the two imaging software.
\index{MIJ@MIJ  \see{Interoperability,}}MIJ also allows MATLAB to
access all built-in functions of ImageJ as well as third-party ImageJ
plugins. The developers provide more information on the \href{http://bigwww.epfl.ch/sage/soft/mij/}{MIJ}
and \href{http://www.mathworks.com/matlabcentral/fileexchange/32344-hardware-accelerated-3d-viewer-for-matlab}{Matlab File Exchange}
websites. \nameref{sub:Fiji-intro} features \code{\href{http://fiji.sc/wiki/index.php/Miji}{Miji.m}},
which makes even more convenient to use the libraries and functions
provided by Fiji's components from within Matlab.
\end{description}

\calso{\href{http://imagej.nih.gov/ij/links.html}{ImageJ related links},
list of \href{http://developer.imagej.net/category/web-links/related-imaging-software}{related imaging software}
on the \nameref{sub:ImageJ2intro} website}


\subsection{ImageJ2\label{sub:ImageJ2intro}}

\href{http://imagejdev.org/}{ImageJDev} is a \href{http://imagejdev.org/funding}{federally funded},
\href{http://imagejdev.org/collaborators}{multi-institution} project
dedicated to the development of the next-generation version of ImageJ:
``\index{ImageJ2}ImageJ2''. \index{ImageJ2}\index{ImageDev@ImageDev \see{ImageJ2,}}ImageJ2
is a complete rewrite of ImageJ, that includes the current, stable
version ImageJ (``ImageJ1'') with a compatibility layer so that
old-style plugins and macros can run the same as they currently do
in ImageJ1. Below is a \href{http://imagejdev.org/aims}{summary}
of the \href{http://imagejdev.org/}{ImageJDev} project aims: 
\begin{itemize}
\item To create the next generation version of ImageJ and improve its core
architecture based on the needs of the community.
\item To ensure ImageJ remains useful and relevant to the broadest possible
community, maintaining backwards compatibility with ImageJ1 as close
to 100\% as possible.
\item Expand functionality by interfacing ImageJ with existing open-source
programs.
\item To lead ImageJ development with a clear vision, avoiding duplication
of efforts
\item To provide a central online resource for ImageJ: program downloads,
a plugin repository, developer resources and more.
\end{itemize}
Be sure to follow the ImageJ2 \href{http://imagejdev.org/recent_changes}{project news}
and the \href{http://imagejdev.org/blog}{ImageDev blog} for updates
on this exciting project.


\section{Getting Help\label{sec:Help-Resources}\index{Help resources}}


\subsection{Help on Image Analysis}

\index{Ethics@Ethics \see{Acceptable manipulation,}}\index{Acceptable manipulation}Below
is a list of online resources (in no particular order) related to
image processing and scientific image analysis, complementing the
list of \href{http://imagej.nih.gov/ij/links.html}{external resources on the IJ web site}. 


\subsubsection*{Ethics in Scientific Image Processing}
\begin{itemize}
\item \href{http://www.ori.dhhs.gov/education/products/RIandImages/default.html}{Online learning Tool for Research Integrity and Image Processing}\\
This website, created by the \href{http://ori.dhhs.gov/}{Office of Research Integrity},
explains what is appropriate in image processing in science and what
is not.
\item \href{http://swehsc.pharmacy.arizona.edu/exppath/micro/digimage_ethics.php}{Digital Imaging: Ethics (at the Cellular Imaging Facily Core, SEHSC)}\\
This website, compiled by Douglas Cromey at the University of Alabama
-- Birmingham, discusses thoroughly the topic of digital imaging ethics.
It is recommended for all scientists. The website contains links to
several external resources, including:\vspace{-8pt}


\begin{enumerate}
\item \href{http://www.jcb.org/cgi/reprint/166/1/11}{What's in a picture? The temptation of image manipulation}
(2004) M Rossner and K M Yamada, J Cell Biology 166(1):11--15, doi:10.1083/jcb.200406019
\item \href{http://www.nature.com/nature/journal/v439/n7079/full/439891b.html}{Not picture-perfect}
(2006), Nature 439, 891--892, doi:10.1038/439891b.
\end{enumerate}
\end{itemize}

\subsubsection*{Scientific Image Processing\index{Image processing (help)}\label{sub:IP-Resources}}
\begin{itemize}
\item \href{http://fiji.sc/wiki/index.php/IP_Principles}{What you need to know about scientific image processing}\\
Simple and clear, this \nameref{sub:Fiji-intro} webpage explains
basic aspects of scientific image processing.
\item \href{http://www.imagingbook.com}{imagingbook.com}\\
Web site of \emph{Digital Image Processing: An Algorithmic Introduction
using Java} by Wilhelm Burger and Mark Burge \cite{Burger:2008p14082}.
This technical book provides a modern, self-contained, introduction
to digital image processing techniques. Numerous complete Java implementations
are provided, all of which work within ImageJ.
\item \href{http://homepages.inf.ed.ac.uk/rbf/HIPR2/}{Hypermedia Image Processing Reference (HIPR2)}\\
Developed at the Department of Artificial Intelligence in the University
of Edinburgh, provides on-line reference and tutorial information
on a wide range of image processing operations.
\item \href{https://ifn.mpi-cbg.de/wiki/ifn/index.php/Imaging_Facility_Network}{IFN wikipage}\\
The Imaging Facility Network (IFN) in Biopolis Dresden provides access
to advanced microscopy systems and image processing. The website hosts
high quality \href{https://ifn.mpi-cbg.de/wiki/ifn/index.php/Teaching_Material}{teaching material}
and useful links to external resources.
\item \href{http://www.stereology.info/}{stereology.info} \\
Stereology Information for the Biological Sciences, designed to introduce
both basic and advanced concepts in the field of stereology.
\end{itemize}

\calso{ImageJ Related Publications on page \pageref{sec:IJ-related-pub}}


\subsection{Help on ImageJ\label{sub:Getting-Help}}

Below is a list of the ImageJ \index{Help resources}help resources
that complement this guide (\emph{see} \nameref{sec:Guide-Formats}).
Specific documentation on advanced uses of ImageJ (macro programming,
plugin development, etc.) is discussed in \nameref{sec:Extending-ImageJ}.
\begin{enumerate}
\item The ImageJ \href{http://imagej.nih.gov/ij/docs/}{online documentation pages}\\
Can be accessed via the \textsf{\userinterface{\textsf{Help\lyxarrow{}\nameref{sub:Documentation...}}}}
command.
\item The \index{Fiji}\nameref{sub:Fiji-intro} webpage:\\
\url{http://fiji.sc/}
\item The ImageJ Information and Documentation Portal (ImageJ wikipage):\\
\href{http://imagejdocu.tudor.lu/doku.php}{http://imagejdocu.tudor.lu/doku.php}
\item Video \index{Tutorials}tutorials on the ImageJ Documentation Portal
and the Fiji YouTube channel:\\
\url{http://imagejdocu.tudor.lu/doku.php?id=video:start&s[]=video}
and \url{http://www.youtube.com/user/fijichannel}. New ImageJ users
will probably profit from \href{http://imagejdocu.tudor.lu/doku.php?id=video:beginner_help:imagej_beginner_s_tutorial}{Christine Labno's video tutorial}.
\item The \index{MBF ImageJ}ImageJ for Microscopy manual\\
\url{http://www.macbiophotonics.ca/imagej/}
\item Several online documents, most of them listed at:\\
\url{http://imagej.nih.gov/ij/links.html} and \url{http://imagej.nih.gov/ij/docs/examples/}
\item Mailing lists:\index{Mailing lists@Mailing lists \see{Help resources,}}

\begin{enumerate}
\item \textbf{ImageJ} --- \url{http://imagej.nih.gov/ij/list.html}\\
General user and developer discussion about ImageJ. Can be accessed
via the \textsf{\userinterface{\textsf{Help\lyxarrow{}\nameref{sub:List-Archives...}}}}
command. This list is also mirrored at \href{http://imagej.1557.n6.nabble.com/}{Nabble}
and \href{http://dir.gmane.org/gmane.comp.java.imagej}{Gmane}. You
may find it easier to search and browse the list archives on these
mirrors. Specially useful are the \href{feed://rss.gmane.org/topics/excerpts/gmane.comp.java.imagej}{RSS feeds}
and the\emph{ \href{http://news.gmane.org/gmane.comp.java.imagej}{frames and threads}}
view provided by Gmane.
\item \textbf{Fiji users }--- \url{http://groups.google.com/group/fiji-users}\\
For user discussion specific to \nameref{sub:Fiji-intro} (rather
than core ImageJ).
\item \textbf{Fiji-devel} --- \url{http://groups.google.com/group/fiji-devel}\\
For developer discussion specific to Fiji.
\item \textbf{ImageJ-devel} --- \url{http://imagejdev.org/mailman/listinfo/imagej-devel}\\
For communication and coordination of the ImageJDev project.
\item \textbf{Dedicated mailing lists} for ImageJ related projects\\
Described at \url{http://imagejdev.org/mailing-lists} .
\end{enumerate}
\end{enumerate}

\subsubsection*{Using Mailing-lists}

If you are having problems with ImageJ, you should inquiry about them
in the appropriated \index{Help resources}list. The ImageJ mailing
list is an unmoderated forum subscribed by a knowledgeable worldwide
user community with $\thickapprox$2000 advanced users and developers.
To have your questions promptly answered you should consider the following:
\begin{enumerate}
\item Read the documentation files (described earlier in this section) before
posting. Because there will always be a natural lag between the implementation
of key features and their documentation it may be wise to check briefly
the ImageJ news website (\textsf{\userinterface{\textsf{Help\lyxarrow{}\nameref{sub:ImageJ-News...}}}}).
\item Look up the mailing list archives (\textsf{\userinterface{\textsf{Help\lyxarrow{}\nameref{sub:List-Archives...}}}}).
Most of your questions may already been answered.
\item If you think you are facing a \index{Bug (reporting)@Bug (reporting) \seealso{Debug,}}bug
try to upgrade to the latest version of ImageJ (\textsf{\userinterface{\textsf{Help\lyxarrow{}\nameref{sub:Update-ImageJ...}}}}).
You should also check if you are running the latest version of the
Java Virtual Machine for your operating system. Detailed instructions
on how to submit a bug report are found at \url{http://imagej.nih.gov/ij/docs/faqs.html#bug}.
\item Remember that in most cases you can find answers within your own ImageJ
installation without even connecting to the internet since the heuristics
for finding commands or writing macros have been significantly improved
in later versions (\emph{see} \nameref{sec:Finding-Commands} and
\nameref{sec:Extending-ImageJ}).
\item As with any other mailing list, you should always follow basic \href{http://en.wikipedia.org/wiki/Netiquette}{netiquette},
namely:

\begin{enumerate}
\item Use descriptive subject lines -- \emph{Re: Problem with Image>Set
Scale command }is much more effective than a general\emph{ Re: Problem.}
\item Stay on topic -- Do not post off-topic messages, unrelated to the
message thread.
\item Be careful when sending attachments -- Refrain from attaching large
files. Use, e.g., a \href{http://en.wikipedia.org/wiki/File_hosting_service\#Comparison_of_notable_file_hosting_services}{file hosting service}
instead.
\item Edit replies -- You should include only the minimum content that is
necessary to provide a logical flow from the question to the answer,
i.e., quote only as much as absolutely necessary and relevant.\end{enumerate}
\end{enumerate}

