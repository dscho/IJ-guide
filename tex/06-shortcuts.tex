
\part{Keyboard Shortcuts\label{sec:Keyboard-Shortcuts}}

The following table summarizes the keyboard \index{Shortcuts}shortcuts
built into ImageJ. You can create additional shortcuts, or override
built-in ones, by \href{http://imagej.nih.gov/ij/developer/macro/macros.html\#shortcuts}{creating simple macros}
and adding them to the \filenameref{StartupMacros.txt}. You can also
assign a function key to a menu command using \textsf{\userinterface{\textsf{Plugins\lyxarrow{}Shortcuts\lyxarrow{}}\nameref{sub:Create-Shortcuts...}}}. 

Several of these shortcuts accept key modifiers as described in \nameref{sec:Key-Modifiers}.
Also note that, except when using the \nameref{sec:Text-Tool}, you
do not need to hold down the control key to use a keyboard shortcut,
unless \emph{Require control key for shortcuts }in \textsf{\userinterface{\textsf{Edit\lyxarrow{}Options\lyxarrow{}\nameref{sub:Misc...}}}}
is checked.


\calso{\nameref{sub:Using-Shortcuts}, \nameref{sec:Finding-Commands},
\filenameref{\href{http://imagej.nih.gov/ij/macros/KeyboardShortcuts.txt}{KeyboardShortcuts.txt}}
macro (demonstrating how assign shortcuts to custom macros), \nameref{sub:Tools-shortcuts}}

\begingroup
\small
\renewcommand{\tabcolsep}{4pt}%6pt default
\renewcommand\arraystretch{1.08} % arraystretch is a factor!

\noindent %
\begin{longtable}{l>{\raggedright}p{2cm}l}
\caption[List of ImageJ\,\protect\theguideversion{} built-in shortcuts]{\textbf{List of ImageJ\,\protect\theguideversion{} built-in shortcuts.}
This table can be obtained within ImageJ using the \textsf{\protect\userinterface{\textsf{Plugins\lyxarrow{}Shortcuts\lyxarrow{}\nameref{sub:List-Shortcuts...}}}}
command.}
\\%\endfirsthead
\midrule 
\multicolumn{1}{c}{Command\,/\,Operation} & \multicolumn{1}{>{\centering}p{2cm}}{Shortcut} & \multicolumn{1}{c}{Description}\tabularnewline
\midrule
\endhead
\addlinespace
\multicolumn{3}{l}{\userinterface{File\lyxarrow{}}}\tabularnewline\addlinespace[-3pt]
\midrule
\quad{}\userinterface{New\lyxarrow{}\nameref{sub:Image...[n]}} & \mykeystroke{N} & Create new image or stack\tabularnewline
\quad{}\userinterface{New\lyxarrow{}\nameref{sub:TextWindow[N]}} & \noindent \mykeystroke{Shift} \mykeystroke{N} & Create new text window \tabularnewline
\quad{}\userinterface{New\lyxarrow{}\nameref{sub:SystemClipboard[V]}} & \mykeystroke{Shift} \mykeystroke{V} & Create image from system clipboard \tabularnewline
\quad{}\userinterface{\nameref{sub:Open...}} & \mykeystroke{O} & Open file (any format recognized by ImageJ)\tabularnewline
\quad{}\userinterface{\nameref{sub:OpenNext[O]}} & \mykeystroke{Shift} \mykeystroke{O} & Open next image in folder \tabularnewline
\quad{}\userinterface{\nameref{sub:OpenSamples}Blobs (25K)} & \mykeystroke{Shift} \mykeystroke{B} & Opens the \emph{Blobs.gif} example image \tabularnewline
\quad{}\userinterface{\nameref{sub:Close[w]}} & \mykeystroke{W} & Close the active window \tabularnewline
\quad{}\userinterface{\nameref{sub:Save[s]}} & \mykeystroke{S} & Save active image in Tiff format \tabularnewline
\quad{}\userinterface{\nameref{sub:Revert[r]}} & \mykeystroke{R} & Revert to saved version of image\tabularnewline
\quad{}\userinterface{\nameref{sub:Print...[p]}} & \mykeystroke{P} & Print active image \tabularnewline
\addlinespace
\multicolumn{3}{l}{\userinterface{Edit\lyxarrow{}}}\tabularnewline\addlinespace[-3pt]
\midrule
\quad{}\userinterface{\nameref{sub:Undo-[z]}} & \mykeystroke{Z} & Undo last operation \tabularnewline
\quad{}\userinterface{\nameref{sub:Cut[x]}} & \mykeystroke{X} & Copy selection to internal clipboard and clear\tabularnewline
\quad{}\userinterface{\nameref{sub:Copy[c]}} & \mykeystroke{C} & Copy selection to internal clipboard \tabularnewline
\quad{}\userinterface{\nameref{sub:Paste[v]}} & \mykeystroke{V} & Paste contents of internal clipboard \tabularnewline
\quad{}\userinterface{\nameref{sub:Clear}} & \mykeystroke{Backspace} & Erase selection to background color \tabularnewline
\quad{}\userinterface{\nameref{sub:Fill-[f]}} & \mykeystroke{F} & Fill selection in foreground color \tabularnewline
\quad{}\userinterface{\nameref{sub:Draw-[d]}} & \mykeystroke{D} & Draw selection\tabularnewline
\quad{}\userinterface{\nameref{sub:Invert-[I]}} & \mykeystroke{Shift} \mykeystroke{I} & Invert image or selection\tabularnewline
\quad{}\userinterface{Selection\lyxarrow{}\nameref{sub:Select-All-[a]}} & \mykeystroke{A} & Select entire image\tabularnewline
\quad{}\userinterface{Selection\lyxarrow{}\nameref{sub:Select-None-[A]}} & \mykeystroke{Shift} \mykeystroke{A} & Remove selection\tabularnewline
\quad{}\userinterface{Selection\lyxarrow{}\nameref{sub:Restore-Selection-[E]}} & \mykeystroke{Shift} \mykeystroke{E} & Restore previous selection\tabularnewline
\quad{}\userinterface{Selection\lyxarrow{}\nameref{sub:Properties...}} & \mykeystroke{Y} & Defines selection properties\tabularnewline
\quad{}\userinterface{Selection\lyxarrow{}\nameref{sub:Add-to-Manager}} & \mykeystroke{T} & Add selection to ROI Manager \tabularnewline
\addlinespace
\multicolumn{3}{l}{\userinterface{Image\lyxarrow{}}}\tabularnewline\addlinespace[-3pt]
\midrule
\quad{}\userinterface{Adjust\lyxarrow{}\nameref{sub:Brightness/Contrast...[C]}} & \mykeystroke{Shift} \mykeystroke{C} & Adjust brightness and contrast \tabularnewline
\quad{}\userinterface{Adjust\lyxarrow{}\nameref{sub:Threshold...[T]}} & \mykeystroke{Shift} \mykeystroke{T} & Adjust threshold levels\tabularnewline
\quad{}\userinterface{\nameref{sub:Show-Info...}} & \mykeystroke{I} & Display information about active image\tabularnewline
\quad{}\userinterface{\nameref{sub:Image>Properties...}} & \mykeystroke{Shift} \mykeystroke{P} & Display image properties \tabularnewline
\quad{}\userinterface{Color\lyxarrow{}\nameref{sub:Color-Picker...[K]}} & \mykeystroke{Shift} \mykeystroke{K} & Open Color Picker \tabularnewline
\quad{}\userinterface{Stacks\lyxarrow{}\nameref{sub:Next-Slice-[>]}} & \mykeystroke{>} or \mykeystroke{$\rightarrow$} & Go to next stack slice\tabularnewline
\quad{}\userinterface{Stacks\lyxarrow{}\nameref{sub:Previous-Slice-[>]}} & \mykeystroke{<} or \mykeystroke{$\leftarrow$} & Go to previous stack slice \tabularnewline
\quad{}\userinterface{Stacks\lyxarrow{}\nameref{sub:Reslice...[/]}} & \mykeystroke{/} & Reslice stack\tabularnewline
\quad{}\userinterface{Stacks\lyxarrow{}\nameref{sub:Orthogonal-Views}} & \mykeystroke{Shift} \mykeystroke{H} & Toggle orthogonal view display\tabularnewline
\quad{}\userinterface{Stacks\lyxarrow{}Tools\lyxarrow{}\nameref{sub:Start-Animation}} & \mykeystroke{\textbackslash{}} & Start/stop stack animation\tabularnewline
\quad{}\userinterface{Hyperstacks\lyxarrow{}\nameref{sub:Channels...[Z]}} & \mykeystroke{Shift} \mykeystroke{Z} & Open the `Channels' tool \tabularnewline
\quad{}\nameref{sub:Hyperstacks-Intro} & \mykeystroke{>} or \mykeystroke{$\rightarrow$}  & Next hyperstack channel \tabularnewline
\quad{}\nameref{sub:Hyperstacks-Intro} & \mykeystroke{<} or \mykeystroke{$\leftarrow$} & Previous hyperstack channel \tabularnewline
\quad{}\nameref{sub:Hyperstacks-Intro} & \mykeystroke{Ctrl} \mykeystroke{>} & Next hyperstack slice\tabularnewline
\quad{}\nameref{sub:Hyperstacks-Intro} & \mykeystroke{Ctrl} \mykeystroke{<} & Previous hyperstack slice\tabularnewline
\quad{}\nameref{sub:Hyperstacks-Intro} & \mykeystroke{Alt} \mykeystroke{>} & Next hyperstack frame \tabularnewline
\quad{}\nameref{sub:Hyperstacks-Intro} & \mykeystroke{Alt} \mykeystroke{<} & Previous hyperstack frame \tabularnewline
\quad{}\userinterface{\nameref{sub:Crop-[X]}} & \mykeystroke{Shift} \mykeystroke{X} & Crop active image or selection \tabularnewline
\quad{}\userinterface{\nameref{sub:Duplicate...[D]}} & \mykeystroke{Shift} \mykeystroke{D} & Duplicate active image or selection \tabularnewline
\quad{}\userinterface{\nameref{sub:Scale...[E]}} & \mykeystroke{E} & Scale image or selection \tabularnewline
\quad{}\userinterface{Zoom\lyxarrow{}\nameref{sub:ZoomIn}} &  \mykeystroke{$+$} or \mykeystroke{$\uparrow$} & Make image larger \tabularnewline
\quad{}\userinterface{Zoom\lyxarrow{}\nameref{sub:ZoomOut}} & \mykeystroke{$-$} or \mykeystroke{$\downarrow$} & Make image smaller \tabularnewline
\quad{}\userinterface{Zoom\lyxarrow{}\nameref{sub:Original-Scale-[4]}} & \mykeystroke{4} & Revert to original zoom level \tabularnewline
\quad{}\userinterface{Zoom\lyxarrow{}\nameref{sub:View-100=000025-[5]}} & \mykeystroke{5} & Zoom to 1:1\tabularnewline
\quad{}\userinterface{Overlay\lyxarrow{}\nameref{sub:Add-Selection...[b]}} & \mykeystroke{B} & Adds active selection to image overlay\tabularnewline
\addlinespace
\multicolumn{3}{l}{\userinterface{Process\lyxarrow{}}}\tabularnewline\addlinespace[-3pt]
\midrule
\quad{}\userinterface{\nameref{sub:Smooth-[S]}} & \mykeystroke{Shift} \mykeystroke{S} & 3$\times$3 unweighted smoothing \tabularnewline
\quad{}\userinterface{\nameref{sub:Repeat-Command-[R]}} & \mykeystroke{Shift} \mykeystroke{R} & Repeat previous command\tabularnewline
\addlinespace
\multicolumn{3}{l}{\userinterface{Analyze\lyxarrow{}}}\tabularnewline\addlinespace[-3pt]
\midrule
\quad{}\userinterface{\nameref{sub:Measure...[m]}} & \mykeystroke{M} & Display statistics of active image\,/\,selection\tabularnewline
\quad{}\userinterface{\nameref{sub:Histogram}} & \mykeystroke{H} & Display histogram of active image\,/\,selection\tabularnewline
\quad{}\userinterface{\nameref{sub:Plot-Profile-[k]}} & \mykeystroke{K} & Display density profile plot of active selection\tabularnewline
\quad{}\userinterface{\nameref{sub:Gels}Select First Lane} & \mykeystroke{1} & Select first gel lane\tabularnewline
\quad{}\userinterface{\nameref{sub:Gels}Select Next Lane} & \mykeystroke{2} & Select next gel lane\tabularnewline
\addlinespace
\multicolumn{3}{l}{\userinterface{Plugins\lyxarrow{}}}\tabularnewline\addlinespace[-3pt]
\midrule
\quad{}\userinterface{Utilities\lyxarrow{}\nameref{sub:Control-Panel...}} & \mykeystroke{Shift} \mykeystroke{U} & Open Control Panel\tabularnewline
\quad{}\userinterface{Utilities\lyxarrow{}\nameref{sub:Capture-Screen-[g]}} & \mykeystroke{Shift} \mykeystroke{G} & Grab screenshot (with \mykeystroke{Ctrl} if a dialog box is active)\tabularnewline
\quad{}\userinterface{Utilities\lyxarrow{}\nameref{sub:Command-Finder}} & \mykeystroke{L} & List, find and launch commands\tabularnewline
\addlinespace
\multicolumn{3}{l}{\userinterface{Window\lyxarrow{}}}\tabularnewline\addlinespace[-3pt]
\midrule
\quad{}\userinterface{\nameref{sub:ShowAll}} & \mykeystroke{{]}} & Make all windows visible\tabularnewline
\quad{}\userinterface{\nameref{sub:PutBehind}} & \mykeystroke{Tab} & Switch to next image window\tabularnewline
\quad{}\nameref{fig:The-ImageJ-window} & \mykeystroke{Enter} & Bring ImageJ window to front\tabularnewline\addlinespace
\bottomrule
\end{longtable}


\section{Key Modifiers\label{sec:Key-Modifiers}\index{Modifier keys}}


\subsection{Alt Key Modifications}
\begin{description}
\item [{\textsf{File\lyxarrow{}\nameref{sub:OpenNext[O]}}}] Open previous
\item [{\textsf{File\lyxarrow{}\nameref{sub:Revert[r]}}}] Skip dialog
prompt
\item [{\textsf{Edit\lyxarrow{}\nameref{sub:Copy[c]}}}] Copy to system
clipboard
\item [{\textsf{Image\lyxarrow{}Color\lyxarrow{}\nameref{sub:Split-Channels}}}] Keep
original image
\item [{\textsf{Image\lyxarrow{}Stacks\lyxarrow{}\nameref{sub:Add-Slice}}}] Insert
before current slice 
\item [{\textsf{Image\lyxarrow{}Stacks\lyxarrow{}\nameref{sub:Next-Slice-[>]}}}] Skip
ten slices 
\item [{\textsf{Image\lyxarrow{}Stacks\lyxarrow{}\nameref{sub:Previous-Slice-[>]}}}] Skip
ten slices 
\item [{\textsf{Image\lyxarrow{}Stacks\lyxarrow{}\nameref{sub:Start-Animation}}}] Show
options dialog
\item [{\textsf{Image\lyxarrow{}\nameref{sub:Duplicate...[D]}}}] Skip
dialog prompt
\item [{\textsf{Image\lyxarrow{}Overlay\lyxarrow{}\nameref{sub:Add-Selection...[b]}}}] Show
options dialog
\item [{\textsf{Process\lyxarrow{}\nameref{sub:Enhance-Contrast}}}] Do
classic histogram equalization
\item [{\textsf{Analyze\lyxarrow{}\nameref{sub:Histogram}}}] Show dialog
prompt
\item [{\textsf{Analyze\lyxarrow{}\nameref{sub:Plot-Profile-[k]}}}] For
rectangular selections, generate row average plot. For wide straight
lines, display rotated contents 
\item [{\textsf{Analyze\lyxarrow{}\nameref{sub:Gels}Select\ First\ Lane}}] Assumes
lanes are horizontal
\item [{\textsf{Analyze\lyxarrow{}Tools\lyxarrow{}\nameref{sub:Analyze-Line-Graph}}}] Show
intermediate image 
\item [{\textsf{Analyze\lyxarrow{}Tools\lyxarrow{}\nameref{fig:The-ROI-Manager}\ (}\textsf{\emph{Add}}\textsf{)}}] Name
and add selection 
\item [{\textsf{Plugins\lyxarrow{}Utilities\lyxarrow{}\nameref{sub:ImageJ-Properties...}}}] List
all Java properties
\end{description}
\nameref{sec:Area-selection-tools} Subtract current selection from
the previous one

\nameref{sub:Rectangular-Selection-Tool}\ and\ \nameref{sub:Oval-Selection-Tool}
Current aspect ratio is maintained while resizing

\nameref{sub:Straight-Line-Selection} Keeps the line length fixed
while moving either end of the line. Forces the two points that define
the line to have integer coordinate values when creating a line on
a zoomed image 

\nameref{sub:Segmented-Line-Selection}\ and\ \nameref{sub:Polygon-Selection-Tool}
Alt-clicking on a node deletes it 

\nameref{sec:Point-Tool} Alt-clicking on a point deletes it 

\nameref{sec:Color-Picker} Alt-clicking on an image `picks-up'
background color 

All\textsf{\ }\nameref{sec:IJ-Tools} Show location and size in pixels
rather than calibrated units


\subsection{Shift Key Modifications}
\begin{description}
\item [{\userinterface{Image\lyxarrow{}Adjust\lyxarrow{}\nameref{sub:Threshold...[T]}}}] Adjusting
\emph{Min} also adjusts \emph{Max}
\item [{\userinterface{Image\lyxarrow{}Adjust\lyxarrow{}\nameref{sub:Brightness/Contrast...[C]}}}] Apply
adjustments to all channels of a composite image
\item [{Installed\ \nameref{sub:Macros-ExtendingIJ}\ and\ \nameref{sub:Scripts}}] Open
instead of run
\end{description}
\nameref{sub:Rectangular-Selection-Tool}\ and\ \nameref{sub:Oval-Selection-Tool}
Forces 1:1 aspect ratio 

\nameref{sec:Area-selection-tools} Add selection to previous one

\nameref{sub:Polygon-Selection-Tool} Shift-clicking on a node duplicates
it

\nameref{sub:Straight-Line-Selection} Forces line to be horizontal
or vertical

\nameref{sub:Segmented-Line-Selection} Shift-clicking on a node duplicates
it

\nameref{sec:Point-Tool} Shift-clicking adds points (\nameref{sec:Multi-point-Tool}
behavior)

\nameref{sec:Magnifying-Glass} Shift-clicking and dragging runs \textsf{\userinterface{Image\lyxarrow{}Zoom\lyxarrow{}\nameref{sub:ZoomToSelection}}}


\calso{\nameref{sub:Manipulating-ROIs}, \nameref{sec:IJ-Tools}}


\subsection{Ctrl (or Cmd) Key Modifications}

\nameref{sub:Rectangular-Selection-Tool}\ and\ \nameref{sub:Oval-Selection-Tool}
Selection is resized around its center 

\nameref{sub:Straight-Line-Selection} Line is rotated/resized around
its center


\calso{\nameref{sub:Manipulating-ROIs}, \nameref{sec:IJ-Tools}}


\subsection{Space Bar}
\begin{description}
\item [{Any\ Tool}] Switch to the \nameref{sec:Scrolling-Tool}
\end{description}

\subsection{Arrow Keys\label{Arrow-Keys}}
\begin{description}
\item [{Moving\ \nameref{sec:Selections-Intro}}] The four arrow keys
move selection outlines one pixel at a time
\item [{Resizing\ \nameref{sec:Selections-Intro}}] Rectangular and oval
selections are resized by holding \mykeystroke{Alt} while using the
arrow keys
\item [{\nameref{sub:Stacks-Intro}\ Navigation}] The \mykeystroke{$\leftarrow$}
and \mykeystroke{$\rightarrow$} keys substitute for \mykeystroke{<}
and \mykeystroke{>} for moving through a stack. If there is a selection,
you must also hold \mykeystroke{Shift}
\item [{\nameref{sub:Hyperstacks-Intro}\ Navigation}] The \mykeystroke{$\leftarrow$}
and \mykeystroke{$\rightarrow$} keys change the channel. Hold \mykeystroke{Ctrl}
to move through the slices and \mykeystroke{Alt} to move through
the frames
\item [{Zooming}] The \mykeystroke{$\uparrow$} and \mykeystroke{$\downarrow$}
keys zoom the image in and out. If there is a selection, you must
also hold either \mykeystroke{Shift} or \mykeystroke{Ctrl}
\end{description}

\calso{\nameref{sub:Manipulating-ROIs}, \userinterface{\nameref{sub:Zoom}}}

\endgroup


\section{Toolbar Shortcuts\label{sub:Tools-shortcuts}}

Keyboard shortcuts cannot be used directly to activate tools in the
ImageJ \nameref{sub:Toolbar} (with the exception of the \nameref{sec:Magnifying-Glass}
and the \nameref{sec:Scrolling-Tool}). However, shortcuts can be
assigned to macros that use the \code{\href{http://imagej.nih.gov/ij/developer/macro/functions.html\#setTool}{setTool()}}
macro function. 

The set of macros listed below (taken from \filenameref{\href{http://imagej.nih.gov/ij/macros/ToolShortcuts.txt}{ToolShortcuts}})
exemplify  how to assign the function keys \mykeystroke{F1} through
\mykeystroke{F12} to some of the most commonly used \nameref{sec:IJ-Tools}.
Once copied to the \filenameref{ImageJ/macros/StartupMacros.txt}
file, they will be automatically installed at startup.

\begin{lstlisting}[caption={[Assigning Keyboard Shortcuts to ImageJ Tools]Assigning Keyboard Shortcuts to ImageJ Tools},label={lis:toolsShrtct1},showstringspaces=false,tabsize=4]
/* These macros allow tools to be selected by pressing function keys. Add them to ImageJ/macros/StartupMacros.txt and they will be automatically installed when ImageJ starts. */

 macro "Rectangle [f1]" {setTool("rectangle")}
 macro "Elliptical [f2]" {setTool("elliptical")}
 macro "Brush [f3]" {setTool("brush")}
 macro "Polygon [f4]" {setTool("polygon")}
 macro "Freehand [f5]" {setTool("freehand")}
 macro "Straight Line [f6]" {setTool("line")}
 macro "Segmented Line [f7]" {setTool("polyline")}
 macro "Arrow [f8]" {setTool("arrow")}
 macro "Angle [f9]" {setTool("angle")}
 macro "Multi-point [f10]" {setTool("multipoint")}
 macro "Wand [f11]" {setTool("wand")}
 macro "Magnifying Glass [f12]" {setTool("zoom")}
\end{lstlisting}


This approach, however, requires the user to memorize a large number
of shortcuts. In addition, it may be difficult to assign so many hot-keys
without conflicting with previously defined ones (\emph{see} \userinterface{Plugins\lyxarrow{}\nameref{sub:Shortcuts}}).
An alternative way to control the toolbar using the keyboard is to
create macros that progressively activate tools from a predefined
sequence. The next example demonstrates such strategy. It is composed
of two macros activated by \mykeystroke{F1} and \mykeystroke{F2}
that iterate through the toolbar from left to right (forward cycle)
and right to left (reverse cycle). 

\begin{lstlisting}[caption={[Cycling Through the Toolbar Using Keyboard Shortcuts]Cycling Through the Toolbar Using Keyboard
Shortcuts},label={lis:toolsShrtct2},showstringspaces=false,tabsize=4]
/* These two macros loop through the tools listed in an array using "F1" and "F2" as keyboard shortcuts (forward and reverse cycling). */

 var index;
 var tools = newArray("rectangle", "roundrect", "oval", "ellipse", "brush", "polygon", "freehand", "line", "freeline", "polyline", "arrow", "wand", "dropper", "angle", "point", "multipoint", "text");

 macro "Cycle Tools Fwd [F1]" {
  setTool(tools[index++]);
  if (index==tools.length) index = 0; 
 } 
   
 macro "Cycle Tools Rwd [F2]" {
  if (index<0) index = tools.length-1;
  setTool(tools[index--]);
 }
\end{lstlisting}


A tool can be defined either by its name or by its position in the
toolbar using \code{\href{http://imagej.nih.gov/ij/developer/macro/functions.html\#setTool}{setTool(id)}},
which allows assigning keyboard shortcuts to \nameref{sec:CustomToolsAndToolsets}
and items loaded by the \nameref{sec:ToolSwitcher} (e.g., \code{setTool(21);}
activates whatever tool has been installed on the last slot of the
toolbar). It is also possible to temporarily activate a tool. The
macro below (taken from the \filenameref{\href{http://imagej.nih.gov/ij/macros/toolsets/Rename\%20and\%20Save\%20ROI\%20Sets.txt}{Rename and Save ROI Sets}}
toolset), activates the \nameref{sec:Color-Picker} when \mykeystroke{F3}
is pressed, but restores the previously active tool as soon as the
mouse is released.

\begin{lstlisting}[caption={Temporary Activation of a Tool},label={lis:toolsShrtct3},showstringspaces=false,tabsize=4]
 macro "Pick Color Once [F3]" {
   tool = IJ.getToolName; 
   setTool("dropper");
   while (true) {
       getCursorLoc(x, y, z, flags);
       if (flags&16!=0)
           { setTool(tool); exit; }
   }
 }
\end{lstlisting}

